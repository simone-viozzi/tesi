% LTeX: enabled=false
%Compile with: pdflatex Tesi.tex && bibtex Tesi.aux && pdflatex Tesi.tex && pdflatex Tesi.tex

\documentclass[11pt,oneside,openright,a4paper]{book}

\usepackage[a-1b]{pdfx} %PDF/A
\usepackage[utf8]{inputenc}
\usepackage[italian]{babel}
\usepackage{amsmath}
\usepackage{amsfonts}
\usepackage{hyperref} 
\usepackage{amssymb}
\usepackage{pdflscape}
\usepackage{chngcntr}
\usepackage{makeidx}
\usepackage{rotating}
\usepackage[utf8]{inputenc}
\usepackage{array}
\usepackage{subcaption}
\usepackage{datetime}
\usepackage{xcolor}
\usepackage{colortbl}
\usepackage{rotating}
\usepackage{tabto}
\usepackage{qrcode}
\usepackage{pstricks} %Per codice QR
\usepackage{pst-barcode} %Per codice QR
\usepackage{eurosym}
\usepackage{graphicx}
\usepackage{fancyhdr}
\usepackage{fancyvrb}
\usepackage{lmodern}
\usepackage{xurl} %per spezzare gli url inseriti con \url
\usepackage{svg}
\svgpath{{../imgs/}}
\usepackage[left=2.5cm,right=2.5cm,top=3cm,bottom=3cm]{geometry}
\author{Alessandro Illuminati}
\hypersetup{
   % bookmarks=true,         % show bookmarks bar?
    %unicode=false,          % non-Latin characters in Acrobat’s bookmarks
   % pdftoolbar=true,        % show Acrobat’s toolbar?
   % pdfmenubar=true,        % show Acrobat’s menu?
   % pdffitwindow=false,     % window fit to page when opened
   % pdfstartview={FitH},    % fits the width of the page to the window
   % pdftitle={My title},    % title
   % pdfauthor={Author},     % author
   % pdfsubject={Subject},   % subject of the document
   % pdfcreator={Creator},   % creator of the document
   % pdfproducer={Producer}, % producer of the document
   % pdfkeywords={keyword1, key2, key3}, % list of keywords
   % pdfnewwindow=true,      % links in new PDF window
    colorlinks=true,       % false: boxed links; true: colored links
    linkcolor=black,          % color of internal links (change box color with linkbordercolor)
    %citecolor=green,        % color of links to bibliography
    %filecolor=magenta,      % color of file links
    urlcolor=blue           % color of external links
}



%______PREFAZIONE__________

\begin{document}

\frontmatter
\begin{singlespace}
  \subfile{contenuti/1.frontespizio}
  %\subfile{contenuti/0.bianca}
  %\input{contenuti/dedica}
  %\input{contenuti/0.bianca}
  \begin{doublespace}
    \subfile{contenuti/2.prefazione}
  \end{doublespace}

  %___________INDICE_CAPITOLI__________
  \tableofcontents 

  %___________INDICE_FIGURE_______
  \listoffigures

\end{singlespace}

%___________CONTENUTI_______________________________
\mainmatter

\subfile{contenuti/3.introduzione_cap_1}
%   - stack iso / osi -> cos'e' un'ip
%   - openvpn
%   - openwrt


\subfile{contenuti/4.capitolo_2}
% overview
%   - overview dell'architettura da ottenere (50%)
%   - specifiche dei componenti (40%)


\subfile{contenuti/5.capitolo_3}
% server
%   - ca e chiavi varie
%   - installazione e config di openvpn
%   - test della config con singolo host
%   - automazione creazione delle config

\subfile{contenuti/6.capitolo_4}
% router
%   - configurazione openwrt
%   - configurazione firewall

%\subfile{contenuti/capitolo_4}
% testing della config
%   - test di connessione tra router e server
%   - test di connessione tra router e host vpn
%   - test della connessione tra host sotto il router e host della vpn

%\subfile{contenuti/capitolo_5}
% multi-istanza
%   - multi-istanza openvpn
%   - problemi di sicurezza relativi

%\subfile{contenuti/capitolo_6}
%   - multi-istanza con virtualizzazione del server su docker

%\input{contenuti/conclusione}


%___________BIBLIOGRAFIA____________________

\begin{singlespace}
		
  \clearpage
  %\noappendicestocpagenum
  %\addappheadtotoc
  %\appendixpage
  %\subfile{sections/appendices}

  \bibliographystyle{abbrv} %Available styles: plain abbrv plainnat, others on: https://www.overleaf.com/learn/latex/Bibtex_bibliography_styles
  \bibliography{refs/references} %File.bib styles: https://www.overleaf.com/learn/latex/Bibliography_management_with_natbib
  
\end{singlespace}


%___________________________________________________________

%\input{contenuti/ringraziamenti}

\end{document}