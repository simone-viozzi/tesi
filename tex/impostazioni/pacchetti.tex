% LTeX: enabled=false

\usepackage[a4paper,lmargin=30mm, textwidth=160mm, tmargin=20mm, textheight=257mm]{geometry}

\usepackage[a-1b]{pdfx} %PDF/A
%One-and-a-half spacing shall be used, except for...
\usepackage[onehalfspacing]{setspace}
\usepackage[utf8]{inputenc}
\usepackage[T1]{fontenc}
\usepackage[italian]{babel}
\usepackage{amsfonts}
\usepackage{hyperref}%\hyperref[appendix:test]{A}
\hypersetup{
	colorlinks	= true,
	allcolors	= {blue},
	linktocpage	= true,
	citecolor=blue,
}
\urlstyle{same}
%\usepackage{url} %per spezzare gli url inseriti con \url
\usepackage{amssymb}
\usepackage{chngcntr}
\usepackage{makeidx}
\usepackage[bottom]{footmisc}
\usepackage[titletoc]{appendix}
\usepackage[justification=centering]{caption}
\usepackage{subcaption}
\usepackage{datetime}
\usepackage{xcolor}
\usepackage{colortbl}
\usepackage{tabto}
\usepackage{eurosym}
\usepackage{graphicx,subfiles,natbib,tocbibind,amsmath,array,pdflscape}
\usepackage{fancyvrb}
\usepackage{lmodern}
\usepackage{bibentry}
\nobibliography*
\usepackage{float}
\usepackage{lipsum,blindtext,multirow,rotating,dcolumn}
\usepackage{tcolorbox}
\usepackage{cprotect}
\usepackage{svg}
\svgpath{{../imgs/}}

\definecolor{light-gray}{gray}{0.95}
\newcommand{\code}[1]{\colorbox{light-gray}{\Verb+#1+}}

\usepackage{minted}

\tcbuselibrary{minted,skins}


\newtcblisting[auto counter, number within=section]{bashcode}[2]{
    listing engine=minted,
    colback=light-gray,
    colframe=black!50,
    listing only,
    minted style=colorful,
    minted language=bash,
    minted options={
		linenos=true,
		numbersep=3mm,
		texcl=true,
        %#2,
		breaklines,
		texcomments,
		fontsize=\footnotesize
	},
    overlay={
        \begin{tcbclipinterior}
            \fill[black!25] (frame.south west)
              rectangle ([xshift=5mm]frame.north west);
        \end{tcbclipinterior}
    },
    coltitle=white,
    boxrule=0.5mm,
    left=5mm,
    title={\large \textit{#1}\hfill code: \thetcbcounter},
    label=code:#2,
    enhanced jigsaw,
}


%***************************************************
%\setlength{\parindent}{10ex} will be indented 10ex (an "ex" equals the lenght of the "x" in the current font)
\setlength{\parindent}{0pt}%no paragraph indent

\setlength{\parskip}{0.8em}%Paragraph spacing
\setlength{\baselineskip}{1.2em}%Line spacing



\usepackage{fancyhdr}

\pagestyle{fancy}
\fancyhf{}
\fancyfoot[R]{\thepage}
\renewcommand{\headrulewidth}{0pt}

\fancypagestyle{plain}{
	\fancyhf{}
	\fancyfoot[R]{\thepage}
	\renewcommand{\headrulewidth}{0pt}
}


\usepackage{etoolbox}

\makeatletter

%... indented quotations and footnotes, where single spacing may be used.
\AtBeginEnvironment{quotation}{\singlespacing}
\AtBeginEnvironment{quote}{\singlespacing}

%\listoftables, \listofalgorithms, and \listoffigures chapter titles.
\newcommand{\thechaptername}{}
\newcounter{chapter@last@figure}
\newcounter{chapter@last@table}
\newcounter{chapter@last@algorithm}

\renewcommand{\chaptermark}[1]{
	\markboth{#1}{}
	\renewcommand{\thechaptername}{#1}
}

\pretocmd{\caption}{
	\ifnum\pdfstrcmp{\@captype}{figure}=0
		\ifnum\value{chapter}=\value{chapter@last@figure}
		\else
			\addtocontents{lof}
			{\protect\numberline{\bfseries\thechapter\quad\thechaptername}}
		\fi
	\fi
	
	\ifnum\pdfstrcmp{\@captype}{table}=0
		\ifnum\value{chapter}=\value{chapter@last@table}
		\else
			\addtocontents{lot}
			{\protect\numberline{\bfseries\thechapter\quad\thechaptername}}
		\fi
	\fi
	
	\ifnum\pdfstrcmp{\@captype}{algorithm}=0
		\ifnum\value{chapter}=\value{chapter@last@algorithm}
		\else
			\addtocontents{loa}
			{\protect\numberline{\bfseries\thechapter\quad\thechaptername}}
		\fi
	\fi
\expandafter\setcounter\expandafter{chapter@last@\@captype}{\value{chapter}}
}{}{failure}


\let\svthefootnote\thefootnote

\newcommand\blankfootnote[1]{%
	\let\thefootnote\relax\footnotetext{#1}%
	\let\thefootnote\svthefootnote%
}
\let\svfootnote\footnote
\renewcommand\footnote[2][?]{%
	\if\relax#1\relax%
	\blankfootnote{#2}%
	\else%
	\if?#1\svfootnote{#2}\else\svfootnote[#1]{#2}\fi%
	\fi
}

\makeatother

\let\it\textit
\let\bf\textbf

\newif\ifdebug
\debugtrue

\newcommand{\todo}[1][""]{
	\ifdebug
		\if #1\relax
			{\footnotesize \textcolor{red}{\textbf{\textit{TODO}}}}
		\else
			{\footnotesize \textcolor{red}{\textbf{\textit{TODO:}} #1}}
		\fi
	\fi
}

\newcommand{\workinprogress}{
	\ifdebug
		{\footnotesize \textcolor{blue}{\textbf{work-in-progress}}}
	\fi
}

\newcommand{\ok}{
	\ifdebug
		{\footnotesize \textcolor{green}{\textbf{OK}}}
	\fi
}	



%***************************************************
\DeclareUnicodeCharacter{2502}{\mbox{\kern.23em %│
	\vrule height 2.2 ex depth 1 ex width .4 pt}}
\DeclareUnicodeCharacter{251C}{\mbox{\kern.23em %├
	\vrule height 2.2 ex depth 1 ex width .4 pt \vrule height 2.2 pt depth -1.8 pt width .23em}}
\DeclareUnicodeCharacter{2500}{\mbox{ %─
	\vrule height 2.2 pt depth -1.8 pt width .5em}}
\DeclareUnicodeCharacter{2514}{\mbox{\kern.23em %└
	\vrule height 2.2 ex depth -1.8 pt width .4 pt \vrule height 2.2 pt depth -1.8 pt width .23em}}
\DeclareUnicodeCharacter{25CF}{$\bullet$}
%***************************************************

\author{Simone Viozzi}


