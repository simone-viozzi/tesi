\chapter{Conclusione workinprogress}

\begin{comment}
Conclusioni

Al momento di scrivere le conclusioni spesso si rimette mano anche all'introduzione poiché sono due sezioni tra loro collegate. Non occorre ripetere quindi le conclusioni già inserite eventualmente nei vari capitoli, ma piuttosto inserire una breve analisi del lavoro svolto, delle problematiche affrontate e delle metodologie per risolverle e indicare possibili futuri sviluppi.

scaletta
    1. sta tesi è una guida implementativa
    2. sta architettura è stata implementata su apparati reali
    3. per la tesi l'architettura è stata simulata
        1. quindi ci stanno delle differenze, es ho usato openwrt stock
    4. evoluzione futura multi-istanza usando una pki perogni processo vpn


\end{comment}

L'architettura proposta in questo elaborato è stata effettivamente implementata su apparati reali durante il tirocinio in collaborazione con l'azienda Esse-ti. Per rendere più completo il prodotto che vogliono vendere è necessario modificare la configurazione per consentire che più clienti, ogniuno con il proprio Router 4G, possano usufrire della VPN. Si deve però stare attenti a mantenere l'isolamento tra i clienti, per questo motivo è necessario garantire che un cliente assegnato ad una determinata VPN non possa connettersi a quelle degli altri.