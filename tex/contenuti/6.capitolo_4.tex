
\chapter{Configurazione Router}

\section{Overview della configurazione}

In questo capitolo andremo a connettere il router 4g alla rete vpn e aggiungere le opportune regole in modo che il traffico proveniente dall'host domotico sia indirizzato verso la VPN

\newsavebox{\myimage}
\begin{figure}[H]
    \centering
    \savebox{\myimage}{
        \includegraphics[width=0.45\linewidth]{immagini/diag-router_real}
    }
    \begin{subfigure}{0.4\textwidth}
        \centering
        \usebox{\myimage}
        \caption{Diagramma di stato del router}
        \label{fig:diag-router}
    \end{subfigure}
    \hfill 
    \begin{subfigure}{0.5\textwidth}
        \centering
        \raisebox{\dimexpr.5\ht\myimage-.5\height\relax}{
            \includegraphics[width=1\linewidth]{immagini/diag-router_goal}
        }
        \caption{Diagramma di stato del router}
        \label{fig:diag-router1}
    \end{subfigure}
\end{figure}


\section{Introduzione a Luci}

L'interfaccia grafica dovrebbe essere gia' installata e raggiungibile, in caso contrario puo' essere installata e configurata seguendo la guida ufficiale di \it{OpenWrt} \cite{install-luci}.

% TODO la figura non sta al posto suo
\begin{figure}[H]
    \centering

    \begin{subfigure}{0.5\textwidth}
        \centering
        \includegraphics[height=0.6\linewidth]{immagini/LuCI_login}
        \caption{Login page}
        \label{fig:luci-login}
    \end{subfigure}%
    \hfill
    \begin{subfigure}{0.5\textwidth}
        \centering
        \includegraphics[height=0.6\linewidth]{immagini/LuCI_status}
        \caption{Status page}
        \label{fig:luci-status}
    \end{subfigure}%

    \begin{subfigure}{0.5\textwidth}
        \centering
        \includegraphics[height=0.6\linewidth]{immagini/LuCI_graphs}
        \caption{Graphs page}
        \label{fig:luci-graphs}
    \end{subfigure}%
    \hfill
    \begin{subfigure}{0.5\textwidth}
        \centering
        \includegraphics[height=0.6\linewidth]{immagini/LuCI_interfaces}
        \caption{Interfaces page}
        \label{fig:luci-graphs}
    \end{subfigure}%
    \caption{Interfaccia grafica LuCI}
\end{figure}

Le credenziali di default sono \code{Username:root} \code{Password:root}, come mostrato in fig. \ref{fig:luci-login}.

L'homepage, fig. \ref{fig:luci-status}, mostra un riepilogo dello stato del router, ad esempio sono presenti: informazioni sull'hardware, informazioni sulla memoria e storage, sono presenti inoltre informazioni riassuntive sulle interfacce di rete e sul \it{DHCP}.

L'interfaccia e' estensiva e permette di configurare quasi ogni aspetto del funzionamento del router, compreso il firewall, il \it{DHCP}, i processi in esecuzione, etc. 

\section{Accesso ssh al Router}

Oltre all'interfaccia grafica \it{LuCI} si puo' accedere al router tramite ssh.

\begin{bashcode}{Router 4g}{}
BusyBox v1.35.0 (2022-04-24 21:09:51 UTC) built-in shell (ash)
    _______                     ________        __
    |       |.-----.-----.-----.|  |  |  |.----.|  |_
    |   -   ||  _  |  -__|     ||  |  |  ||   _||   _|
    |_______||   __|_____|__|__||________||__|  |____|
            |__| W I R E L E S S   F R E E D O M
    -----------------------------------------------------
    OpenWrt SNAPSHOT, r19521-46980294f6
    -----------------------------------------------------
\end{bashcode}

Non esiste un account utente quindi viene effettuato il login come root.

\section{Creazione della configurazione e test}

Si devono seguire gli step descritti in sezione \ref{sec:client_keys}, quindi creare la certificate request e firmarla nel \it{server CA}. Per poi usare lo script creato in sezione \ref{sec:script_client} per costruire il file di configurazione:

\begin{bashcode}{Server}{}
$ ./make_config.sh router
\end{bashcode}

Dopodiché si deve spostare il file \code{router.ovpn} dal \it{Server} al \it{Router 4g}, supponiamo di averlo copiato nella cartella \code{/configs}. 

Di default non e' presente \it{OpenVPN} nel \it{Router 4g}, lo si puo' installare con:

\begin{bashcode}{Router 4g}{}
$ opkg update
$ opkg install openvpn
$ opkg install luci-app-openvpn
\end{bashcode}

Ora possiamo avviare il client openvpn:

\begin{bashcode}{Router 4g}{}
$ openvpn --config /configs/router.ovpn
2022-04-29 17:26:37 OpenVPN 2.5.6 x86_64-openwrt-linux-gnu [SSL (mbed TLS)] [LZ4] [EPOLL] [MH/PKTINFO] [AEAD]
[...]
2022-04-29 17:26:37 VERIFY EKU OK
2022-04-29 17:26:37 VERIFY OK: depth=0, CN=server
2022-04-29 17:26:37 Control Channel: TLSv1.2, cipher TLS-ECDHE-RSA-WITH-AES-256-GCM-SHA384, 2048 bit key
2022-04-29 17:26:37 [server] Peer Connection Initiated with [AF_INET]10.0.4.2:1194
2022-04-29 17:26:37 net_addr_ptp_v4_add: 10.8.0.10 peer 10.8.0.9 dev tun0
2022-04-29 17:26:37 Initialization Sequence Completed
\end{bashcode}

Se il file di configurazione e' stato creato correttamente si vedra' il messaggio \\\code{Initialization Sequence Completed}.

Comparira' inoltre l'interfaccia \code{tun0} a cui e' stato assegnato l'indirizzo \code{10.8.0.10}.

Per abilitare l'autostart di openvpn per il router si deve, per prima cosa, modificare il file \code{/etc/config/openvpn} in modo che faccia riferimento alla config corretta:

\begin{bashcode}{Router 4g}{}
$ vim /etc/config/openvpn
20  option config /configs/router.ovpn
\end{bashcode}

Ora possiamo abilitarla usando luci:

\begin{figure}[H]
    \centering
    \includegraphics[width=0.6\textwidth]{immagini/LuCI_vpn}
    \caption{Configurazione della VPN tramite LuCI}
    \label{fig:luci-vpn}
\end{figure}

Si deve mettere il check su \it{enabled} e premere start, per poi salvare le modifiche. In questo modo il router si connettera' automaticamente alla VPN anche se venisse riavviato.

\section{Abilitazione del Client-to-Client nel server OpenVPN}

In questo momento i client della VPN, \it{client1} e \it{router}, possono comunicare tra loro, ma lo fanno passando per la network stack del \it{server}. Infatti: 

\begin{bashcode}{Router 4g}{}
$ ping -c2 10.8.0.2                  # client1
PING 10.8.0.2 (10.8.0.2): 56 data bytes
64 bytes from 10.8.0.2: seq=0 ttl=63 time=0.519 ms
64 bytes from 10.8.0.2: seq=1 ttl=63 time=0.501 ms

--- 10.8.0.2 ping statistics ---
2 packets transmitted, 2 packets received, 0% packet loss
round-trip min/avg/max = 0.501/0.510/0.519 ms
\end{bashcode}

Dal server possiamo vedere i pacchetti con \it{tcpdump}:

\begin{bashcode}{Server}{}
$ sudo tcpdump -i tun0
listening on tun0, link-type RAW (Raw IP), snapshot length 262144 bytes
16:20:50.791063 IP 10.8.0.3 > 10.8.0.2: ICMP echo request, id 1759, seq 0, length 64
16:20:50.791098 IP 10.8.0.3 > 10.8.0.2: ICMP echo request, id 1759, seq 0, length 64
16:20:50.791273 IP 10.8.0.2 > 10.8.0.3: ICMP echo reply, id 1759, seq 0, length 64
16:20:50.791285 IP 10.8.0.2 > 10.8.0.3: ICMP echo reply, id 1759, seq 0, length 64
16:20:51.791153 IP 10.8.0.3 > 10.8.0.2: ICMP echo request, id 1759, seq 1, length 64
16:20:51.791174 IP 10.8.0.3 > 10.8.0.2: ICMP echo request, id 1759, seq 1, length 64
16:20:51.791365 IP 10.8.0.2 > 10.8.0.3: ICMP echo reply, id 1759, seq 1, length 64
16:20:51.791374 IP 10.8.0.2 > 10.8.0.3: ICMP echo reply, id 1759, seq 1, length 64
\end{bashcode}

Si vede che ogni richiesta viene duplicata, la prima e' in entrata sulla network stack del \it{server} e la seconda in uscita. 

Per evitare questo traffico possiamo abilitare l'opzione \code{client-to-client} nel file di configurazione del server. In questo modo il layer openvpn effettuera' direttamente il forwarding tra i client della vpn \cite{client-to-client}.

\begin{bashcode}{Server}{}
$ vim /etc/openvpn/server/server.conf
209  client-to-client
$ sudo systemctl restart openvpn-server@server.service
\end{bashcode}

Possiamo quindi rieseguire gli stessi test fatti sopra:

\begin{bashcode}{Router 4g}{}
$ ping -c2 10.8.0.2
PING 10.8.0.2 (10.8.0.2): 56 data bytes
64 bytes from 10.8.0.2: seq=0 ttl=64 time=0.351 ms
64 bytes from 10.8.0.2: seq=1 ttl=64 time=0.307 ms

--- 10.8.0.2 ping statistics ---
2 packets transmitted, 2 packets received, 0% packet loss
round-trip min/avg/max = 0.307/0.329/0.351 ms
\end{bashcode}

Ma questa volta la network stack del \it{server} non vede nessun pacchetto:

\begin{bashcode}{Server}{}
$ sudo tcpdump -i tun0
listening on tun0, link-type RAW (Raw IP), snapshot length 262144 bytes
\end{bashcode}