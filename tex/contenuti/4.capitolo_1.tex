

\chapter{Overview dell'architettura e delle componenti utilizzate}
\setlength{\parskip}{1em}
\setlength{\parindent}{0em}
\renewcommand{\baselinestretch}{1.15}

\label{ch:1}

\section{Obbiettivo da ottenere}

In una collaborazione tra il Dipartimento di Ingegneria dell'Informazione e l'azienda \textbf{Esse-ti S.R.L.} ci \`e stato esposto un progetto che consiste nel:
\begin{itemize}
    \item fornire a dei clienti un router 4G, su cui possono essere connessi vari dispositivi, ad es. di tipo domotico.
    \item rendere questi dispositivi accessibili ai clienti attraverso internet 
\end{itemize}

\begin{figure}[ht]
	\centering
	\includesvg[width=250px]{immagini/goal}
	\caption{Schema concettuale dell'obbiettivo da raggiungere}

	%TODO manca label \label{fig:modello_a_strati}
	
\end{figure}

Si vede subito che, cos\`i come \`e rappresentata, non \`e realizzabile a meno che il cliente non abbia un'IP pubblico e la sua macchina venga configurata opportunamente. Questo per\`o non \`e possibile nel caso generale, quindi per risolvere efficacemente questa topologia si deve necessariamente introdurre una terza macchina provvista di IP pubblico e che funga da ponte tra il 4G.Router e il cliente.

\begin{figure}[ht]
	\centering
	\includesvg[width=250px]{immagini/real}
	\caption{Schema concettuale dell'architettura che si dovr\`a implementare}

	%TODO manca label \label{fig:modello_a_strati}
	
\end{figure}

%todo rivedi frase
In questo modo si pu\`o configurare una VPN sul server OVH e connettervi sia il 4G.Router che la macchina del cliente. In questo modo l'unica configurazione che il cliente dovr\`a fare \`e l'installazione di un cliente VPN, ci\`o \`e il minimo possibile di configurazione.

\begin{figure}[ht]
	\centering
	\includesvg[width=250px]{immagini/virtual}
	\caption{Schema concettuale dell'architettura che si dovr\`a implementare}

	%TODO manca label \label{fig:modello_a_strati}
	
\end{figure}
