
\chapter{Connessione degli Host domotici alla VPN}
\label{ch:connessione-host-domotici}

\section{Overview della configurazione}

L'ultima parte della configurazione consiste nel rendere disponibile nella VPN la rete locale del \textit{router 4g}, consentendo quindi lo scambio di dati tra gli \textit{host-domotici} e i \textit{Client} della VPN.

Per fare ciò si dovrà configurare il \textit{Router} in modo che effettui il forwarding verso la VPN, e il \textit{Server} in modo che annunci la sottorete del \textit{Router} a tutti gli Host della VPN.

\todo[non sta nel capitolo 2 sta cosa? va bene ripetere?]

Per simulare un host domotico è stata usata una \textit{raspberry pi}, ciò ci consente di poter accedere alla sua shell per effettuare i vari test necessari per confermare che la topologia funzioni correttamente. 

In fig.~\ref{fig:diag2-host_real} possiamo vedere la configurazione iniziale per questo capitolo, cioè la configurazione finale del capitolo \ref{ch:configurazione-router} con l'aggiunta dell'\textit{host domotico} connesso al \textit{Router}. 


\begin{figure}[H]
    \centering
    \savebox{\myimage}{
        \includegraphics[width=0.44\linewidth]{immagini/diag2-host_real}
    }
    \begin{subfigure}{0.44\linewidth}
        \centering
        \usebox{\myimage}
        \caption{Diagramma della configurazione iniziale}
        \label{fig:diag2-host_real}
    \end{subfigure}
    \hfill
    \begin{subfigure}{0.53\linewidth}
        \centering
        \raisebox{\dimexpr.5\ht\myimage-.6\height\relax}{
            \includegraphics[width=1\linewidth]{immagini/diag2-host_virtual}
        }
        \caption{Diagramma finale}
        \label{fig:diag2-host_virtual}
    \end{subfigure}
    \caption{Schemi concettuali della configurazione iniziale e finale per il capitolo \ref{ch:connessione-host-domotici}}
\end{figure}

Una volta conclusa la configurazione si arriverà a una topologia virtuale descritta in fig.~\ref{fig:diag2-host_virtual}, che costituisce inoltre la topologia di obbiettivo per questo elaborato (fig.~\ref{fig:schema_architettura_virtuale}).

\section{Creazione della zona firewall per la VPN}
\label{sec:creazione-firewall-zone-vpn}

Per rendere possibile la comunicazione tra la rete \textit{lan} del \textit{Router} e la VPN è necessario configurare opportunamente il firewall nel \textit{Router}.

Ciò viene fatto direttamente da LuCI, nella sezione firewall si avranno preconfigurate delle zone firewall, in questo caso sono presenti di default 2 zone: \textit{lan} e \textit{wan}:

\begin{figure}[H]
    \centering
    \includegraphics[width=1\linewidth]{immagini/LuCI_firewall_init1}
    \caption{Configurazione di default delle zone del firewall.}
    \label{fig:luci-firewall-init}
\end{figure}

Le 2 zone sono collegate alle relative interfacce: la zona \textit{lan} è collegata all'interfaccia relativa alla rete interna del \textit{Router}; la zona \textit{wan} è collegata all'interfaccia esterna del \textit{Router}. 

Andiamo quindi ad aggiungere una nuova zona, chiamata \textit{vpn}, che successivamente sarà collegata all'interfaccia virtuale della VPN (\textit{tun0}). La zona \textit{vpn} dovrà avere le seguenti opzioni:

\begin{itemize}
    \item Policy di forward: accept
    \item Forward consentito verso la zona \textit{lan}
    \item Forward consentito dalla zona \textit{lan}
\end{itemize}

Possiamo vedere la pagina di configurazione con le opzioni inserite:

\begin{figure}[H]
    \centering
    \includegraphics[width=1\linewidth]{immagini/LuCI_firewall_vpn1}
    \caption{Schermata di aggiunta di una nuova zona firewall.}
    \label{fig:luci-firewall-vpn}
\end{figure}

Dopo aver salvato, la pagina del firewall sarà:

\begin{figure}[H]
    \centering
    \includegraphics[width=1\linewidth]{immagini/LuCI_firewall_end1}
    \caption{Configurazione delle zone firewall dopo l'aggiunta della zona \textit{vpn}.}
    \label{fig:luci-firewall-end}
\end{figure}

\section{Aggiunta dell'interfaccia \textit{tun0} alla zona firewall \textit{vpn}}
\label{sec:aggiunta-interfaccia-tun0-zona-vpn}

Per far effettivamente funzionare la zona firewall aggiunta nella sezione precedente, si deve aggiungere l'interfaccia relativa alla VPN (\textit{tun0}) alla zona \textit{vpn}.

Dato che l'interfaccia \textit{tun0} non è presente in LuCI, la si deve creare usando le opzioni che già conosciamo. Durante la creazione, nella sezione \textit{Firewall Settings}, sarà possibile assegnare l'interfaccia a una zona firewall.

Quindi nella sezione \textit{interfaces}, si deve aggiungere una nuova interfaccia con le seguenti opzioni:

\begin{itemize}
    \item Name: tun0
    \item Proto: static
    \item Device: tun0
    \item ipv4 address: 10.8.0.254
    \item ipv4 netmask: 255.255.255.0
    \item Assign firewall zone: vpn
\end{itemize}

\begin{figure}[H]
    \centering
    \begin{subfigure}{1\linewidth}
        \centering
        \includegraphics[width=1\linewidth]{immagini/LuCI_int_tun0_1}
        \caption{General settings}
        \label{fig:luci-firewall-interfaces}
    \end{subfigure}
    \medskip
    \begin{subfigure}{1\linewidth}
        \centering
        \includegraphics[width=1\linewidth]{immagini/LuCI_int_tun0_2}
        \caption{Firewall settings}
        \label{fig:luci-firewall-interfaces1}
    \end{subfigure}
    \caption{Assegnazione interfaccia \textit{tun0} alla zona firewall \textit{vpn} tramite interfaccia LuCI}
\end{figure}

A questo punto i pacchetti provenienti dalla lan del \it{Router} vengono inoltrati correttamente nella VPN, ma i device nella VPN non hanno una rotta verso la rete interna del router. Quindi in questo momento la comunicazione è monodirezionale.

Possiamo vederlo praticamente con un test:

\begin{itemize}
    \item l'host domotico effettua il ping verso il client
    \item il \textit{Router} è in ascolto sull'interfaccia \textit{tun0} con \code{tcpdump}
\end{itemize}

\begin{bashcode}{Host-domotico}{}
$ ping -c1 10.8.0.2     # client 1
PING 10.8.0.2 (10.8.0.2) 56(84) bytes of data.

--- 10.8.0.2 ping statistics ---
1 packets transmitted, 0 received, 100% packet loss, time 0ms
\end{bashcode}

\begin{bashcode}{Router}{}
$ tcpdump -i tun0
15:33:38.363082 IP 192.168.130.2 > 10.8.0.2: ICMP echo request, id 17, seq 1, length 64
\end{bashcode}

Si vede come il pacchetto ICMP \textit{echo request} venga correttamente inoltrato nella VPN dal \textit{Router}, ma il \code{ping} fallisce poiché che non vi è nessun \textit{echo reply} dal \textit{Client}. Questi problemi verranno risolti prossima sezione.


\section{Modifiche alla configurazione OpenVPN del \textit{Server}}
\label{sec:hosts-openvpn-server}

Per instaurare una comunicazione bidirezionale tra \textit{host-domotico} e \textit{Client} è necessario che:

\begin{enumerate}
    \item il \textit{Server OpenVPN} sia consapevole che il \textit{Router} vuole esporre una sua sottorete verso la VPN
    \item i client della VPN abbiano l'opportuna rotta per raggiungere la sottorete dove si trova l'\textit{host-domotico}
\end{enumerate}

Per il punto 1 è necessario aggiungere l'opzione \code{iroute} nel file \\\code{/etc/openvpn/server/ccd/router}, creato in sezione \ref{sec:static-ip-router}:

\begin{bashcode}{Server}{}
$ vim /etc/openvpn/server/ccd/router
ifconfig-push 10.8.0.254 255.255.255.0
iroute 192.168.130.0 255.255.255.0      # net e netmask della rete lan del router
\end{bashcode}

Per il punto 2 è necessario aggiungere la seguente riga alla configurazione OpenVPN nel \textit{Server}:

\begin{bashcode}{Server}{}
$ vim /etc/openvpn/server/server.conf
168 push "route 192.168.130.0 255.255.255.0"
\end{bashcode}

Dopo aver modificato la configurazione del Server è necessario riavviarlo con \\\code{systemctl restart}.

% TODO da rivedere
In questo modo nella procedura di connessione alla VPN, i client aggiungeranno la rotta verso la sottorete \code{192.168.130.0/24} nella loro tabella di routing. Possiamo verificarlo con: 

\begin{bashcode}{Client}{}
$ ip route
[...]
192.168.130.0/24 via 10.8.0.1 dev tun0
\end{bashcode}

\section{Test e analisi della configurazione \workinprogress}

Ora che la configurazione della topologia è conclusa, raggiungendo l'obbiettivo posto in sezione \ref{sec:overview-goal}, dobbiamo testarla approfonditamente per confermare che tutto funzioni come previsto. 

\todo[da rivedere]

Nei vari test verranno usate le utility \code{ping}, \code{tracepath} e \code{tcpdump}, poste in vari punti della topologia, in modo da valutare il percorso reale e virtuale dei pacchetti.

Supponiamo che il \textit{Client} voglia comunicare con un \textit{host-domotico}, in questo caso la \textit{raspberry pi} che ha IP \code{192.168.130.2}, e che il dato da trasmettere sia un ICMP \textit{echo request}, facente parte di un \code{ping}.

Quindi il \textit{Client} fa partire il ping:

\begin{bashcode}{Client}{}
$ ping -c 1 192.168.130.2
\end{bashcode}

Supponiamo di poter seguire il percorso del pacchetto analizzando gli step principali, in questo caso consideriamo 6 step:

\begin{figure}[H]
    \centering
    \includegraphics[width=0.5\linewidth]{immagini/diag2-test_real}
    \caption{Diagramma di test realizzato con il \textit{Router} e il \textit{Client}}
    \label{fig:diag-test-real}
\end{figure}

\todo[da rileggere mooolto bene]

\begin{enumerate}
    \item Il \textit{Client} (\code{10.8.0.2}) effettua il ping verso l'\textit{host-domotico} (\code{192.168.130.2}). Dato che il \textit{Client} ha la subnet \code{192.168.130.0/24} nella sua tabella di routing (sezione \ref{sec:hosts-openvpn-server}) viene costruito un pacchetto di tipo ICMP \textit{echo request} che ha come IP sorgente \code{10.8.0.2} e come destinazione \code{192.168.130.2}. A questo punto il \it{Client OpenVPN} cifra e incapsula il pacchetto in un ulteriore pacchetto IP (sezione \ref{subsec:openvpn-networking}), che questa volta ha come IP di destinazione \code{51.178.141.119}, cioè l'ip pubblico del \textit{Server}. Il pacchetto può essere quindi trasmesso attraverso internet, esce dal \textit{Client} e raggiunge il \textit{Server}.
    

    \item Il \textit{Server} riceve il pacchetto, vedendo che si tratta di un pacchetto di tipo OpenVPN lo inoltra al \it{Server OpenVPN} (sezione \ref{sec:network_stack} e \ref{sec:firewall}). Il contenuto del pacchetto viene quindi decifrato e analizzato, il \textit{Server OpenVPN} vede che il pacchetto è destinato nella subnet \code{192.168.130.0/24}, data la configurazione fatta in sezione \ref{sec:hosts-openvpn-server} il \it{Server} sa che deve inoltrarlo al \textit{Router} (\code{10.8.0.254}).
    

    \item Viene costruito un pacchetto IP destinato al NAT, che ha al suo interno il pacchetto cifrato OpenVPN costruito al punto 2. La comunicazione attraverso il NAT è possibile poichè OpenVPN implementa nel suo protocollo una serie di pacchetti \textit{keepalive} tra ongi host e il server. Ciò permette al nat di sapere a quale host della sua rete privata deve inoltrare i pacchetti.
    
    \item Il NAT effettua la traduzione, modificando l'IP e la porta del pacchetto esterno e lo inoltra all'interfaccia \textit{wan} del \textit{Router}.
    

    \item Il \textit{Router} riceve il pacchetto e lo decifra. Vedendo che il pacchetto è destinato alla subnet \code{192.168.130.0/24} lo inoltra alla sua \textit{lan} (sezione \ref{sec:creazione-firewall-zone-vpn} e \ref{sec:aggiunta-interfaccia-tun0-zona-vpn}). Il pacchetto interno viene quindi inoltrato verso l'\textit{host-domotico} (\code{192.168.130.2}).
    
    \item Il \textit{host-domotico} riceve il pacchetto e legge il contenuto, vedendo che si trtta di ICMP \textit{echo request} provvede a costruire una risposta ICMP \textit{echo reply} e la invia al \textit{Client}, invertendo sorgente e destinazione del pacchetto che aveva ricevuto.

\end{enumerate}


\todo[da qual in poi è un disastro!]


\begin{figure}[H]
    \centering
    \includegraphics[width=0.5\linewidth]{immagini/diag2-test_virtual}
    \caption{Diagramma di test virtuale}
    \label{fig:diag-test_virtual}
\end{figure}



Vediamo quindi un ping dal \it{Client} verso l'\textit{host-domotico}, con \it{Server} e \it{Router} in ascolto sull'interfaccia tun0:


\begin{bashcode}{Client}{}
$ ping -c 1 192.168.130.2
PING 192.168.130.2 (192.168.130.2) 56(84) bytes of data.
64 bytes from 192.168.130.2: icmp_seq=1 ttl=63 time=0.643 ms

--- 192.168.130.2 ping statistics ---
1 packets transmitted, 1 received, 0% packet loss, time 0ms
rtt min/avg/max/mdev = 0.643/0.643/0.643/0.000 ms
\end{bashcode}

\begin{bashcode}{Server}{}
$ tcpdump -i tun0
listening on tun0, link-type RAW (Raw IP), snapshot length 262144 bytes
\end{bashcode}

Il \it{Server} non ha visto traffico poiché ha l'opzione \code{client-to-client} abilitata, ciò comporta che il layer vpn effettua il routing senza passare per l'interfaccia di rete del router.

\begin{bashcode}{Router}{}
$ tcpdump -i tun0
listening on tun0, link-type RAW (Raw IP), capture size 262144 bytes
10:44:07.228387 IP 10.8.0.2 > 192.168.130.2: ICMP echo request, id 12, seq 1, length 64
10:44:07.228550 IP 192.168.130.2 > 10.8.0.2: ICMP echo reply, id 12, seq 1, length 64
\end{bashcode}

Il \it{Router} vede il pacchetto e lo instrada verso l'host corretto.

Provando nella direzione inversa si ottiene lo stesso risultato

\begin{bashcode}{Host-domotico}{}
$ ping -c 1 10.8.0.2
PING 10.8.0.2 (10.8.0.2) 56(84) bytes of data.
64 bytes from 10.8.0.2: icmp_seq=1 ttl=63 time=0.428 ms

--- 10.8.0.2 ping statistics ---
1 packets transmitted, 1 received, 0% packet loss, time 0ms
rtt min/avg/max/mdev = 0.428/0.428/0.428/0.000 ms
\end{bashcode}

e il \it{Router} vede:

\begin{bashcode}{Router}{}
$ tcpdump -i tun0
listening on tun0, link-type RAW (Raw IP), capture size 262144 bytes
11:56:10.469901 IP 192.168.130.2 > 10.8.0.2: ICMP echo request, id 14, seq 1, length 64
11:56:10.470230 IP 10.8.0.2 > 192.168.130.2: ICMP echo reply, id 14, seq 1, length 64
\end{bashcode}

% TODO da espandere
Possiamo inoltre usare \code{tracepath} per vedere gli hop:

\begin{bashcode}{Client}{}
$ tracepath 192.168.130.2
1?: [LOCALHOST]                      pmtu 1500
1:  10.8.0.254                       0.471ms
1:  10.8.0.254                       0.473ms
2:  192.168.130.2                    0.567ms reached
    Resume: pmtu 1500 hops 2 back 2
\end{bashcode}

\begin{bashcode}{host-domotico}{}
$ tracepath 10.8.0.2
1?: [LOCALHOST]                      pmtu 1500
1:  router-1                         0.062ms
1:  router-1                         0.068ms
2:  10.8.0.2                         0.592ms reached
    Resume: pmtu 1500 hops 2 back 2
\end{bashcode}

