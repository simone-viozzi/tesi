\clearpage
\phantom{a}
\vfill


\chapter{Prefazione \workinprogress}


\begin{flushleft}




\begin{comment}

Nell'ambito del mio percorso universitario ho avuto modo di approfondire le tematiche relative al mondo delle reti e del networking, a tal proposito grazie alla possibilità offerta dal Dipartimento di Ingegneria dell'Informazione, dal Prof. Ennio Gambi e dall'Ing. Adelmo De Santis ho conseguito con successo la certificazione "\textit{HUAWEI HCIA Routing and Switching}".\\
Successivamente, grazie alle competenze acquisite, ho collaborato con alcuni miei colleghi
per progettare e realizzare una implementazione di una VPN site-to-site attraverso una connessione radiomobile per conto dell'azienda Esse-ti S.r.l.\\
In questo elaborato verranno esposte le principali fasi del
progetto realizzato, ponendo un particolare focus sulle problematiche iniziali affrontate e all'architettura di rete nel cui ambito è stata realizzata la comunicazione tramite un canale sicuro.
\\
\# NOTA:

La prefazione è una breve nota personale riguardate la tua tesi triennale o magistrale. Qui puoi fornire al lettore informazioni inerenti le origini e il contesto della tua tesi.
Inoltre, la prefazione può essere usata anche per ringraziare chiunque abbia aiutato nella stesura dell’elaborato.

% https://www.scribbr.it/struttura-tesi/prefazione-di-una-tesi/


altro esempio da rosss

## sommario

In questo elaborato si esporranno le tecniche impiegate per progettare un algoritmo di
inseguimento traiettoria per un mini drone. Questo algoritmo è stato progettato per
partecipare ad una competizione proposta da MathWorks con il nome ”MathWorks Minidrone
Competition”. Il percorso è formato da una serie di linee di un determinato colore ed un
cerchio, dello stesso colore del tracciato, che determina la fine del percorso.
Il tutto grazie all’elaborazione di segnali video forniti dalla fotocamera che si trova al di sotto del mini-drone.
Verrà utilizzato un drone della Parrot, modello ”Mambo”, insieme ad un modello fornito da
MathWorks in ambiente Simulink.
In primo luogo, si andranno ad evidenziare quali sono i fenomeni fisici che permettono il volo di un drone multi-rotore.
Successivamente verrà fatta un’analisi dei blocchi principali nel modello, ed infine si esporrà l’algoritmo di inseguimento progettato.



es.
La prima domanda che ci siamo posti in questa guida è stata ‘come iniziare una tesi?’.
Ebbene, partiamo col dare un nome alla parte iniziale della tesi: introduzione.

Spesso, nel corso di un lavoro di ricerca approfondito come la tesi di laurea, si tende a cadere in un errore comune, ossia quello di trascurare l’importanza dell’introduzione, apparentemente la parte “più facile” da scrivere.

In realtà non è affatto così: l’introduzione di una tesi è importantissima ed è la parte della tua tesi che viene letta sempre dalla Commissione.
Insieme alla conclusione rappresenta la parte che contribuisce alla definizione della valutazione finale.

L’introduzione della tesi ha come obiettivo quello di introdurre il tuo lavoro, mostrando una panoramica chiara, sintetica ed esaustiva dei contenuti e dell’argomento affrontato.

Al suo interno devono essere citati i seguenti elementi:

    Gli Obiettivi: la finalità che si intendono raggiungere attraverso l’elaborato, ovvero le domande alle quali risponde la tesi
    Le metodologie e gli strumenti utilizzati per realizzare il lavoro

Tornando a come iniziare l’elaborato, possiamo individuare due tipologie principali di introduzione:

    Una tipologia di introduzione della tesi prevede una sorta di sintesi organizzata dei paragrafi, indicando il contenuto a grandi linee e lo sviluppo di questi ultimi. Si tratta di una panoramica generica sul corpo del tuo elaborato, come una sorta di guida alla lettura.
    Nella seconda tipologia di introduzione, invece, puoi scegliere di parlare in modo generico dell’argomento trattato, senza entrare nello specifico del corpo della tesi. Puoi parlare di come hai approcciato alla tematica generale e perché è importante affrontarla, senza dire però il contenuto dei capitoli.

Quale delle due tipologie è la migliore?
Come spesso accade, non esiste una risposta univoca alla domanda. In generale, devi scegliere tu quale approccio tenere, valutando attentamente le due possibilità in base anche al tipo di argomento trattato.



Sommario

Equivale all'abstract presente negli articoli scientifici.

In pochi paragrafi deve esplicitare i contenuti dell'elaborato finale, facendo riferimento alle parole chiave inserite nel titolo.

Il lettore deve capire dal sommario l'ambito di applicazione del progetto svolto, la problematica esaminata, i metodi e gli algoritmi utilizzati per realizzare il progetto e risolvere gli eventuali problemi incontrati, i risultati ottenuti.

Dalla lettura del sommario il lettore deve comprendere se, per i suoi scopi, può essere utile proseguire o meno la lettura, ovvero se gli argomenti descritti rispondono alle sue necessità. Pensiamo, ad esempio, si pensi a un lettore che sia un laureando che si trova ad affrontare un progetto su temi o ambiti di applicazione collegati: vale la pena leggere questo elaborato?



\end{comment}


\begin{comment}

scaletta
prima parte
  sempre stato curioso su come funziona internet
  è un sistema complesso eppure lo usano tutti senza sapere cos'è
  fatto la certificazione HUAWEI per que
seconda parte
  in sto coso ci concentrimao sull'implementazione della vpn
  con vari test del funzionamento e bla bla


\end{comment}

Sono sempre stato curioso sul funzionamento di Internet, una delle rivoluzioni più importanti fatte dall'uomo, eppure multi lo usano con superficialità e pochi ne conoscono il meccanismo interno. 

Nel mio percorso universitario ho avuto l'opportunità di frequentare il corso di \it{Networking} tenuto presso il Dipartimento di Ingegneria dell’Informazione dall’Ing. Adelmo De Santis, e successivamente conseguire la certificazione \textit{HUAWEI HCIA Routing and Switching}. 

Tale percorso mi ha permesso di approfondire molti aspetti sul funzionamento di Internet e come è stato possibile scalare la rete in modo così veloce e semplice. Con le conoscenze acquisite, insieme ad alcuni miei colleghi, abbiamo avuto modo di progettare e realizzare una VPN site-to-site attraverso una connessione radiomobile in un tirocinio in collaborazione con l'azienda \textit{Esse-ti S.r.l.}

Questo elaborato tratterà l'implementazione della VPN seguendo un approccio incrementale, ponendo molta attenzione nei test necessari e nel troubleshooting.





\end{flushleft}



\vfill
\newpage