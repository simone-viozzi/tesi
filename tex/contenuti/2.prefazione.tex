\clearpage
\phantom{a}
\vfill


\chapter{Prefazione}


\begin{flushleft}

\#TODO da riscrivere 

Nell'ambito del mio percorso universitario ho avuto modo di approfondire le tematiche relative al mondo delle reti e del networking, a tal proposito grazie alla possibilità offerta dal Dipartimento di Ingegneria dell'Informazione, dal Prof. Ennio Gambi e dall'Ing. Adelmo De Santis ho conseguito con successo la certificazione "\textit{HUAWEI HCIA Routing and Switching}".\\
Successivamente, grazie alle competenze acquisite, ho collaborato con alcuni miei colleghi
per progettare e realizzare una implementazione di una VPN site-to-site attraverso una connessione radiomobile per conto dell'azienda Esse-ti S.r.l.\\
In questo elaborato verranno esposte le principali fasi del
progetto realizzato, ponendo un particolare focus sulle problematiche iniziali affrontate e all'architettura di rete nel cui ambito è stata realizzata la comunicazione tramite un canale sicuro.
\\
\# NOTA:

La prefazione è una breve nota personale riguardate la tua tesi triennale o magistrale. Qui puoi fornire al lettore informazioni inerenti le origini e il contesto della tua tesi.
Inoltre, la prefazione può essere usata anche per ringraziare chiunque abbia aiutato nella stesura dell’elaborato.
    

\end{flushleft}



\vfill
\newpage