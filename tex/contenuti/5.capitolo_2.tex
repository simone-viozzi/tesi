\chapter{Configurazione del server}
\setlength{\parskip}{1em}
\setlength{\parindent}{0em}
\renewcommand{\baselinestretch}{1.15}

\label{ch:server}

\section{Overview della configurazione e prerequisiti}

In questa sezione andremo a installare e configurare OpenVPN server sulla VPS di OVHCloud.

I pacchetti necessari sono \code{openvpn} e \code{easy-rsa}, che possono essere installati con:

\begin{bashcode}
$ sudo apt-get update
$ sudo apt-get install -y openvpn easy-rsa
\end{bashcode}

E' inoltre necessario avere un editor di testo, ad es. \code{nano} o \code{vim}

\section{Creazione della Public key infrastructure Certificate Authority (PKI CA)}

%TODO mettere intro su che diamine e' la pki https://datatracker.ietf.org/doc/html/rfc5280


%TODO da tradurrre e sistemare
easy-rsa is a CLI utility to build and manage a PKI CA. In laymen's terms, this means to create a root certificate authority, and request and sign certificates, including intermediate CAs and certificate revocation lists (CRL).



La CA puo' essere configurata sulla stessa macchina dove e' stato installato opnevpn, ma cio' e' sconsigliato per motivi di sicurezza, supponiamo quindi di usare un secondo server chiamato \textit{server CA}

La utility \code{easy-rsa} mette a disposizione il comando \code{make-cadir}, che permette di creare una cartella pronta ad ospitare la Certificate Authority.

Andiamo quindi a crearla, nella home ad esempio:

\begin{bashcode}
$ mkdir ~/openvpn-ca
$ ln -s /usr/share/easy-rsa/* ~/openvpn-ca/
$ chmod 700 /home/ubuntu/openvpn-ca/
$ cd openvpn-ca/
$ ./easyrsa init-pki

init-pki complete; you may now create a CA or requests.
Your newly created PKI dir is: /home/ubuntu/openvpn-ca/pki

$ la
easyrsa  openssl-easyrsa.cnf  pki  vars.example  x509-types
\end{bashcode}

Ora si devono personalizzare le variabili \code{vars}, si puo' sia partire da un file vuoto oppure modificare \code{vars.example} per poi rinominarlo \code{vars}.
Andiamo quindi a creare un nuovo file vars:

\begin{bashcode}
$ vim vars
set_var EASYRSA_REQ_COUNTRY  "IT"
set_var EASYRSA_REQ_PROVINCE "MC"
set_var EASYRSA_REQ_CITY     "Recanati"
set_var EASYRSA_REQ_ORG      "Esse-ti"
set_var EASYRSA_REQ_EMAIL    "s.gasparrini@esse-ti.it"
set_var EASYRSA_REQ_OU       "Esse-ti"

set_var EASYRSA_ALGO         "ec"
set_var EASYRSA_DIGEST       "sha512"
\end{bashcode}

Le variabili nel primo blocco determinano i dati che poi verranno registrati nei certificati.

Le ultime 2 sono opzioni di sicurezza, in particolare si setta l'algoritmo di cifratura %TODO add info

A questo punti si deve laciare il comando \code{build-ca} per costruire la CA:

\begin{bashcode}
$ ./easyrsa build-ca

Note: using Easy-RSA configuration from: ./vars

Using SSL: openssl OpenSSL 1.1.1f  31 Mar 2020

Enter New CA Key Passphrase: 
Re-Enter New CA Key Passphrase: 
read EC key
writing EC key

You are about to be asked to enter information that will be incorporated
into your certificate request.
What you are about to enter is what is called a Distinguished Name or a DN.
There are quite a few fields but you can leave some blank
For some fields there will be a default value,
If you enter '.', the field will be left blank.
-----
Common Name (eg: your user, host, or server name) [Easy-RSA CA]:

CA creation complete and you may now import and sign cert requests.
Your new CA certificate file for publishing is at:
/home/ubuntu/openvpn-ca/pki/ca.crt
    
\end{bashcode}

Eseguendo il comando verra' chiesto di inserire una passshare, che verra' usata per criptare la chiave privata appena generata. Il secondo propt e' relativo al nome da dare alla certificazione, in questo caso e' stato lasciato il valore di default \code{Easy-RSA CA}.

Se si legge il file \code{~/openvpn-ca/pki/ca.crt} si vedra' il classico formato da certificato:

\begin{bashcode}
$ cat ca.crt 
-----BEGIN CERTIFICATE-----
MIIB/TCCAYKgAwIBAgIUCYshxSm8eH1mf504HqQqULfVrakwCgYIKoZIzj0EAwQw
[...]
a3Qts307lAO5q49hK8hn5h43w+2MHvaiC0jXtCJp+hvR
-----END CERTIFICATE-----
\end{bashcode}


\section{Configurazione della PKI di OpenVPN}

Il procedimento e' simile al precedente, ma questa volta va eseguito sul server.

Creiamo quindi una cartella per ospitare la PKI, es \code{~/openvpn-pki}, e linkiamo \code{easy-rsa}. Inoltre limitiamo i permessi all'utente non root che stimao usando, in questo caso "ubuntu".

\begin{bashcode}
$ mkdir ~/openvpn-pki
$ ln -s /usr/share/easy-rsa/* ~/openvpn-pki/
$ sudo chown ubuntu ~/openvpn-pki/
$ chmod 700 ~/openvpn-pki/
$ cd ~/openvpn-pki/
\end{bashcode}

Andiamo a creare un file \code{vars}:

\begin{bashcode}
$ vim vars
set_var EASYRSA_ALGO    "ec"
set_var EASYRSA_DIGEST  "sha512"
\end{bashcode}
 
e concludiamo la creazione della PKI con il comando:

\begin{bashcode}
$ ./easyrsa init-pki

Note: using Easy-RSA configuration from: ./vars

init-pki complete; you may now create a CA or requests.
Your newly created PKI dir is: /home/ubuntu/openvpn-pki/pki

\end{bashcode}

%TODO !
Now that your OpenVPN server has all the prerequisites installed, the next step is to generate a private key and Certificate Signing Request (CSR) on your OpenVPN server. After that you’ll transfer the request over to your CA to be signed, creating the required certificate. Once you have a signed certificate, you’ll transfer it back to the OpenVPN server and install it for the server to use.

Come nome e' stato scelto "server":

\begin{bashcode}
$ ./easyrsa gen-req server nopass

Note: using Easy-RSA configuration from: ./vars

Using SSL: openssl OpenSSL 1.1.1f  31 Mar 2020
Generating an EC private key
writing new private key to '/home/ubuntu/openvpn-pki/pki/private/server.key.438W2xM0g9'
-----
You are about to be asked to enter information that will be incorporated
into your certificate request.
What you are about to enter is what is called a Distinguished Name or a DN.
There are quite a few fields but you can leave some blank
For some fields there will be a default value,
If you enter '.', the field will be left blank.
-----
Common Name (eg: your user, host, or server name) [server]:

Keypair and certificate request completed. Your files are:
req: /home/ubuntu/openvpn-pki/pki/reqs/server.req
key: /home/ubuntu/openvpn-pki/pki/private/server.key
    
\end{bashcode}

%TODO!
This will create a private key for the server and a certificate request file called server.req. Copy the server key to the /etc/openvpn/server directory:

\begin{bashcode}
$ sudo cp /home/ubuntu/openvpn-pki/pki/private/server.key /etc/openvpn/server/
\end{bashcode}

Il secondo file creato, \code{server.req}, corrisponde ad una \textit{Certificate Signing Request (CSR)} che va firmata e validata dalla CA. In questo modo ogni client che si fida della CA si fidera' di conseguenza del server OpenVPN %TODO link a web of trust

\section{Firma del certificato opnevpn dalla CA}

Dobbiamo quindi copiare il file \code{server.req} nel \textit{server CA}, possiamo qualunque metodo purche' sia sicuro, ad esempio con \code{scp}:

\begin{bashcode}
$ scp -3 ubuntu@openvpn_server:/home/ubuntu/openvpn-pki/pki/reqs/server.req ubuntu@ca_server:/tmp
\end{bashcode}

Dobbiamo qundi spostarci sul server CA e importare la \textit{certificate request} e firmarlo:

\begin{bashcode}
$ cd ~/openvpn-ca
$ ./easyrsa import-req /tmp/server.req server
$ ./easyrsa sign-req server server
Using configuration from /home/ubuntu/openvpn-ca/pki/safessl-easyrsa.cnf
Check that the request matches the signature
Signature ok
The Subject\'s Distinguished Name is as follows
commonName            :ASN.1 12:'ChangeMe'
Certificate is to be certified until Mar 11 15:50:45 2025 GMT (1080 days)

Write out database with 1 new entries
Data Base Updated
\end{bashcode}

Verra' creato un file in \code{~/openvpn-ca/pki/issued} chiamato \code{server.crt} che conterra' la chiave pubblica che verra' usata dal server openvpn e inoltre la firma della CA.

Ora si devono copiare i file \code{ca.crt} e \code{server.crt} dal \textit{server CA} al \textit{server OpnenVPN}:


\begin{bashcode}
$ scp -3 ubuntu@ca_server:/home/ubuntu/openvpn-ca/pki/issued/server.crt ubuntu@openvpn_server:/tmp
$ scp -3 ubuntu@ca_server:/home/ubuntu/openvpn-ca/pki/ca.crt ubuntu@openvpn_server:/tmp
\end{bashcode}

Possiamo quindi tornare sul \textit{server OpenVPN} e copiare i 2 file da \code{/tmp} a \code{/etc/openvpn/server}:

\begin{bashcode}
$ sudo cp /tmp/server.crt /etc/openvpn/server
$ sudo cp /tmp/ca.crt /etc/openvpn/server
\end{bashcode}

\section{generazione tls-crypt pre-shared key}

Per aumentare ulteriormente la sicurezza del nostro \textit{server OpenVPN} possiamo creare un'ulteriore chiave, che consiste un una chiave preshared che verra' inserita in tutte le configurazioni e serve ad offuscare il certificato in fase di validazione. Quindi in caso di attacco si dovra' conoscere anche questa chiave.

La creazione va fatta sul \textit{server OpenVPN}:

\begin{bashcode}
$ cd ~/openvpn-pki/
$ openvpn --genkey --secret ta.key
\end{bashcode}

il file generato \code{ta.key} dovra' essere copiato nella directory del server openvpn:

\begin{bashcode}
$ sudo cp ta.key /etc/openvpn/server
\end{bashcode}

\section{Generazione delle chiavi per i clients}

Creiamo una cartella nella home che ospitera' le chiavi dei client e le configurazioni openvpn:

\begin{bashcode}
$ mkdir -p ~/client-configs/keys
$ chmod -R 700 ~/client-configs
\end{bashcode}


